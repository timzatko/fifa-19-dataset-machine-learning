% This is samplepaper.tex, a sample chapter demonstrating the
% LLNCS macro package for Springer Computer Science proceedings;
% Version 2.20 of 2017/10/04
%
\documentclass[runningheads]{llncs}
%
\usepackage{graphicx}
% Used for displaying a sample figure. If possible, figure files should
% be included in EPS format.
%
% If you use the hyperref package, please uncomment the following line
% to display URLs in blue roman font according to Springer's eBook style:
% \renewcommand\UrlFont{\color{blue}\rmfamily}

\usepackage{subfig}

\begin{document}
%
% \title{Contribution Title\thanks{Supported by organization x.}}
% \title{Knowledge Discovery - Analysis of players data in FIFA 2019}
\title{OZNAL - Dátová analýza dát hráčov z hry FIFA 19}
%
%\titlerunning{Abbreviated paper title}
% If the paper title is too long for the running head, you can set
% an abbreviated paper title here
%
\author{Timotej Zaťko\inst{1} \and
Tomáš Hoffer\inst{2}}
%
\authorrunning{T. Zaťko, T. Hoffer}
% First names are abbreviated in the running head.
% If there are more than two authors, 'et al.' is used.
%
\institute{Fakulta informatiky a informačných technológií STU v Bratislave
Ilkovičova 2, 842 16 Bratislava 4\\
\email{xzatkot1@stuba.sk}\\
\url{https://www.fiit.stuba.sk/} \and
Fakulta informatiky a informačných technológií STU v Bratislave
Ilkovičova 2, 842 16 Bratislava 4\\
\email{xhoffer@stuba.sk}\\
\url{https://www.fiit.stuba.sk/}}
%
\maketitle              % typeset the header of the contribution
%
\begin{abstract}
Táto práca obsahuje analýzu dát hráčov z hry FIFA 19, opis dát a ich charakteristiky. V tejto práci skúmame vzťahy atribútov na predikovaný atribút trhovej ceny hráča a jeho hernú pozíciu na ihrisku.

\keywords{Analýza dát, FIFA 19, Strojové Učenie, Regresia, Klasifikácia}
\end{abstract}

%
% https://drive.google.com/drive/u/0/folders/1nwkbhB1-NqVBd7KRJBFtTLefISel52dH
% Budeme pisat v anglictine? Ja by som rad ale treba sa spytat cviciaceho...
%

\section{Opis problému a motivácia}

Rozhodli sme sa pre analýzu dát hráčov z hry FIFA 19. V hre sa nachádza veľké množstvo hráčov z rôznych krajín, hrajúcich v rôznych súťažiach. Ich schopnosti v hre by mali odzrkadlovať ich schopnosti z reálneho sveta. Tvorcovia hry sa o to snažia vytvorením herných atribútov, akými sú napríklad rýchlosť šprintu, sila strely, zakončovanie alebo hlavičkovanie. Keďže dokázali vyjadriť tieto schopnosti hráčov číselne (na určitej škále), zaujíma nás či, a ako herná pozícia hráča závisí od týchto atribútov. Ďaľšou zaujímavou otázkou je ako tieto atribúty (a ktoré najviac) vplývajú na trhovú cenu hráča v hre.

\section{Opis dát s charakteristika dát}
 
% TIMO:
 
% - opis dat
% - najskor high level, pocet atributov, pocer kategorickych, pocet numerickych (pred a po predspracovani)
% - spomenut oblasti atributov - skusit ich zadelit do kategorii
%     - hrac ako clovek - meno, narodnost, vyska, hmotnost, vek
%     - hrac ako futbalista - plat, herne atributy (tych je velmi vela - numericke), Work Rate v utoku/defenzivne, pozicia na ihrisku, body type, medzinarodna reputacia, preferovana noha, sila slabsej nohy
%     - hrac a vztah ku klubu - klub, dlzka kontraktu, typ kontraktu, release claause, joined, jersey number
%     - nerelevantne - photo, face...
    
    
% TIMO:
% sekcia ocistenie dat
% - spomenut predspracovanie - ocistenie dat 
%  - konverzie hodnot - financne hodnoty (tisicky, miliony)
%  - konverzie mier - feet to cm, hmotnost
%  - odstranenie znakov. z numerickycha tributov
%  - konverzia dlzky konktraktu (time)

 
Naša dátová sada obsahuje 18207 záznamov - hráčov, ktorý každý z nich má 87 atribútov. Z toho je 42 numerických a 45 kategorických.
 
\subsection{Očistenie dát}
 
Kvôli analýze bolo treba dátovú sadu, očistiť a urobiť predspracovanie niektorých atribútov. Bolo nutné spraviť nasledovné úpravy -- konverzia a očistenie finančných hodnôt (napr. `\$77.5M`, `\$1K`), konverzia mier (napr. `159lbs`, `5'11`), konverzia časových údajov (napr. `Jan 25, 2019`, `2018`) a rozdelenie niektorých atribútov na viac atribútov - niektoré atribúty obsahovali dva numerické atribúty. Taktiež atribút určujúci pozíciu hráča obsahoval až 38 rôznych herných pozícii, preto sme sa rozhodli tento atribút rozšíriť do ďalších dvoch atribútov v ktorých sme podobné pozície spojili čím vznikli dva atribúty s 13 resp. 4 hodnotami. Po týchto úpravách sa nám počet atribútov zmenil na 129 -- 89 numerických a 40 kategorických atribútov.

Toto veľké množstvo atribútov sme zaradili do nasledovných kategórií, uvádzame k nim aj niektoré atribúty. Atribúty sme rozdelili aj na základe toho či jeho hodnotu musela hra nejakým spôsobom odvodiť z reálneho sveta.

\begin{itemize}
\item \textbf{človek} - meno, národnosť, výška, hmotnosť, vek...
\item \textbf{futbalový hráč (reálny svet)} - názov klubu, číslo dresu, dátum príchodu do klubu, dĺžka kontraktu, plat hráča, hosťovský klub, herná pozícia, preferovaná noha
\item \textbf{futbalový hráč (hra)} - trhová hodnota hráča, triky, pracovitosť v útoku/obrane, vhodnosť na určitú špecifickú pozíciu a potenciálny rast (78 atribútov), medzinárodná reputácia
\item \textbf{futbalové schopnosti hráča určené hrou} (celkovo 34 atribútov) - krátke prihrávky, hlavičkovanie, zakončovanie...
\item \textbf{iné atribúty hry} - logo klubu, vlajka (na základe národnosti), typ postavy (tj. typ herného modelu), typ tváre (tj. kvôli herného modelu)
\end{itemize}

\subsection{Analýza chýbajúcich hodnôt}

% TIMO
% - pozreli sme chybajuce hodnoty a vztahy meczi chybajucimi hodnotami
% - ukazat co sme zistili, dat tam ten insight z deepnotu co je, spomenut dendogram

Ako prvé sme sa pozreli na chýbajúce hodnoty v našej dátovej sade. Početnosti sme vizualizovali stĺpcovým diagramom a vzťahy tepelnou mapou a dendrogramom. Zistili sme, že niektorým hráčom chýbajú hodnoty v atribútoch ako sú klub, plat, dĺžka kontraktu, číslo dresu či výkupná klauzula. Tieto hodnoty chýbajú väčšinou spoločne, o čom sme sa presvedčili v dendrograme a tepelnej mape. Je to aj logické, keďže hráč, ktorý nemá klub nemôže poberať plat alebo mať výkupnú klauzulu v kontrakte. Z našich zistení môžeme konštatovať, že ak hráčovi chýbajú hodnoty v atribúte, väčšinou sa nejedná o chybu v úplnosti dátovej sady.

Analýzu sme ďalej realizovali z pohľadu dvoch atribútov, ktoré sa budeme snažiť predikovať - pozíciu hráča (kategorický atribút) a jeho trhovú hodnotu (numerický atribút). Dátová sada celkovo obsahuje 36 herných pozícií. 

\subsection{Analýza z pohľadu pozície hráča} 
Na základe našich futbalových znalostí sme sa rozhodli zoskupiť podobné pozície do skupín, čím sme znížili počet rôznych hodnôt v atribúte `Position` (Fig. \ref{fig:position_grouping}). Môžeme pozorovať, že triedy nie sú vyvážené.

\begin{figure}%
    \centering
    \subfloat[4 herné pozície(`Position (4)`)]{{ 
        \includegraphics[width=4cm]{images/position_grouping_1}
    }}%
    \qquad
    \subfloat[13 herných pozícií (`Position (13)`)]{{
        \includegraphics[width=7cm]{images/position_grouping_2}
    }}%
    \caption{Početnosti nových atribútov `Position (4)` a `Position (13)` po zoskupení podobných hodnôt z atribútu `Position`.}%
    \label{fig:position_grouping}%
\end{figure}

Pozíciu sa môžeme pokúsiť predikovať na základe 34 atribútov definujúcich futbalové schopnosti hráča určené hrou. Vizualizovali sme všetky dvojice týchto atribútov a pomocou "scatter plot" -ov  sme hľadali zhluky jednotlivých pozícií. Ako príklad uvádzame vzťah medzi atribútmi `Shot Power` (sila strely) a `Finishing` (zakončovanie), kde môžeme pozorovať jednotlivé zhluky podľa pozície hráča (Fig. \ref{fig:shot_power_finishing_scatter_plot}).

\begin{figure}%
    \centering
    \subfloat[4 herné pozície (`Position (4)`)]{{ 
        \includegraphics[width=5.5cm]{images/shot_power_finishing_scatter_plot_position_4}
    }}%
    \qquad
    \subfloat[13 henrých pozícií (`Position (13)`)]{{
        \includegraphics[width=5.5cm]{images/shot_power_finishing_scatter_plot_position_13}
    }}%
    \caption{Vzťah medzi atribútmi `Shot Power` a `Finishing`. Pre 4 herné pozície sú zhluky zreteľnejšie ako pre 13 herných pozícií.}%
    \label{fig:shot_power_finishing_scatter_plot}%
\end{figure}

Z prieskumnej analýzy tiež vyplýva, že pre konkrétne pozície hráčov sú typické určité čísla dresov. Pre brankárov (angl. goalkeeper) je typickým číslom dresu číslo \textit{1} pričom toto číslo nemá priredený žiaden hráč na inej pozícii. Pre útočníkov (angl. attack) to je \textit{9}, pre obrancov (angl. defense) sú to čísla \textit{2 - 6} a pre stredopoloarov \textit{7}, \textit{8} a \textit{10}. Tento atribút môže byť veľmi dobrý na klasifikáciu pozície hráča, avšak my sa v prvom rade zamieriame na predikciu z herných atribútov. Atribúty, ako napríklad čislo dresu nám môžu úlohu príliš zjednodušiť.

Ďaľšou zaujímavosťou je, že v našej dátovej sade sa nachádzajú prevažne hráči ktorí preferujú pravú nohu, avšak na pozícii ľavého obrancu výrazne prevládajú hráči s preferovanou ľavou nohou. (Fig. \ref{fig:preferred_foot_counts}).

\begin{figure}[htp]
    \centering
    \includegraphics[height=7cm]{images/preferred_foot_counts}
    \caption{Preferovaná noha hráčov podľa hernej pozície.}
    \label{fig:preferred_foot_counts}
\end{figure}

% \begin{figure}[htp]
%     \centering
%     \includegraphics[height=7cm]{images/value_position_boxenplot}
%     \caption{}
%     \label{fig:value_position_boxenplot}
% \end{figure}

\subsection{Analýza z pohľadu trhovej hodnoty hráča}
Pomocou Pearsonovho korelačného koeficientu sme hľadali korelácie medzi atribútom `Value` a ostatnými numerickými atribútmi. Takmer lineárnu koreláciu (0.99) vykazuje atribút `Release Cause` (Fig. \ref{fig:wage_scatter_plot}). Hráči s vysokou trhovou hodnotou majú v datasete podpísanú zmluvu s vyššou výkupnou klauzulou. Vysokú koreláciu taktiež vykazujú atribúty `Overall` (0.631), `Wage` (0.850) a `International Reputation` (0.656).

\begin{figure}%
    \centering
    \subfloat[Vzťah `Release Cause` voči `Value`.]{{ 
        \includegraphics[width=5cm]{images/release_clause_value_scatter_plot}
    }}%
    \qquad
    \subfloat[Vzťah `Wage` voči `Value`.]{{
        \includegraphics[width=5.5cm]{images/wage_value_scatter_plot}
    }}%
    \caption{Niektoré atribúty vykazujú vysokú mieru korelácie s atribútom `Value`.}%
    \label{fig:wage_scatter_plot}%
\end{figure}


\section{Definovanie úlohy objavovania znalostí}

Rozhodli sme sa, že budeme vykonávať nasledujúce úlohy:
\begin{itemize}
    \item predikcia hernej pozície hráča -- všeobecnej (4 triedy), rozšírenej (13 tried)
    \item predikcia hodnoty hráča (atribút s názvom `Value`)
\end{itemize}

Obe tieto úlohy budeme realizovať z atribútov určujúcich herné shopnosti hráča a následne zo všetkých atribútov. Výsledky porovnáme a očakávame, že model natrénovaný zo všetkých atribútov bude výrazne lepší, keďže niektoré atribúty výrazne ovplyvňujú predikovanú premennú a to -- Release Clause - Value; Jersey Number - Position. 

\section{Predpokladaný scenár riešenia (problémy)}

Predpokladáme, že bude potrebné vykonať nasledovné úlohy:
\begin{itemize}
    \item predspracovanie kategorických hodnôt (tj. one-hot encoding)
    \item normalizácia dát
    \item odstránenie odľahlých pozorovaní
    \item trénovanie modelu
\end{itemize}

Trénovanie modelu zahŕňa výber atribútov (angl. feature selection) a výber a trénovanie modelu. Na úlohu predikcie hodnoty hráča budeme pravdepodobne používať lineárnu regresiu/neurónovú sieť a na určenie hernej pozície hráča rozhodovací strom / náhodný les / SVM / neurónovú sieť.

% \bibliographystyle{splncs04}
% \bibliography{references}

\end{document}
